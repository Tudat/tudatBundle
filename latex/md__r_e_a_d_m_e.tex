The Tudat bundle contains Tudat and a number of external libraries in a modular fashion. The TU Delft Astrodynamics Toolbox (Tudat) is a set of C++ software libraries for simulating various astrodynamics applications, for more info see \href{http://tudat.tudelft.nl}{\tt the Tudat website}.

\subsection*{Instructions}

This document contains minimal information on how to build the Tudat bundle. For more information see documentation below.

\subsubsection*{Building the project}


\begin{DoxyEnumerate}
\item Clone the repository on your computer (or fork and then clone) \begin{DoxyVerb} git clone https://github.com/tudat/tudatBundle.git
\end{DoxyVerb}

\item Enter the new directory \begin{DoxyVerb} cd tudatBundle
\end{DoxyVerb}

\item Checkout all the submodules (optionally you can clone only the necessary submodules) \begin{DoxyVerb} git submodule update --init --recursive
\end{DoxyVerb}

\item Make a new build directory and enter \begin{DoxyVerb} mkdir build && cd build
\end{DoxyVerb}

\item Initiate C\+Make for the project \begin{DoxyVerb} cmake ../
\end{DoxyVerb}

\item Build the project \begin{DoxyVerb} make
\end{DoxyVerb}

\end{DoxyEnumerate}

Note\+: it can happen that due to dependencies between the submodules the {\ttfamily make} process quits with an error (after building S\+P\+I\+CE or J\+S\+O\+N\+C\+PP, for instance). In this case, please repeat steps 5 and 6 once or twice.

\label{_switching_apps}%
 \subsubsection*{Switching on/off libraries and applications}

By default only the S\+P\+I\+CE library is build. You can enable/disable which libraries are build by manipulating special use switches\+:


\begin{DoxyEnumerate}
\item Each library has such a switch, these are\+: {\ttfamily U\+S\+E\+\_\+\+C\+S\+P\+I\+CE}, {\ttfamily U\+S\+E\+\_\+\+J\+S\+O\+N\+C\+PP}, {\ttfamily U\+S\+E\+\_\+\+N\+R\+L\+M\+S\+I\+SE} and {\ttfamily U\+S\+E\+\_\+\+P\+A\+G\+MO}.
\item You can turn on or off such a switch as an argument to C\+Make. For instance, the following will disable S\+P\+I\+CE, but enable the nrlmsise-\/00 atmopshere model\+: \begin{DoxyVerb} cmake ../ -DUSE_CSPICE=0 -DUSE_NRLMSISE=1
\end{DoxyVerb}

\end{DoxyEnumerate}

Detailed control and to specification of detailed build options for external libraries (like building of examples and tests) can be achieved by editing {\ttfamily C\+Make\+Lists.\+txt}.

\subsubsection*{Creating your own applications}


\begin{DoxyEnumerate}
\item Copy the {\ttfamily template\+Application} from {\ttfamily tudat\+Example\+Applications} to tudat\+Applications and give it your own name\+: \begin{DoxyVerb} cp -R tudatExampleApplications/templateApplication tudatApplications/myApplication
 cd tudatApplications/myApplication
 mv TemplateApplication MyApplication
\end{DoxyVerb}

\item Add your project to the top-\/level {\ttfamily C\+Make\+Lists.\+txt}, like so\+: \begin{DoxyVerb} add_subdirectory("${PROJECTROOT}/tudatApplications/myApplication/MyApplication")
\end{DoxyVerb}

\item Re-\/run cmake and make commands.
\end{DoxyEnumerate}

\subsection*{Documentation}


\begin{DoxyItemize}
\item \href{http://tudat.tudelft.nl/projects/tudat/wiki}{\tt Tudat wiki}
\item \href{http://tudat.tudelft.nl/Doxygen/Tudat/docs/index.html}{\tt Doxygen} 
\end{DoxyItemize}